\let\negmedspace\undefined
\let\negthickspace\undefined
%\RequirePackage{amsmath}
\documentclass[journal,12pt,twocolumn]{IEEEtran}
%
% \usepackage{setspace}
 \usepackage{gensymb}
  \usepackage[misc]{ifsym}
%\doublespacing
 \usepackage{polynom}
%\singlespacing
%\usepackage{silence}
%Disable all warnings issued by latex starting with "You have..."
%\usepackage{graphicx}
\usepackage{amssymb}
%\usepackage{relsize}
\usepackage[cmex10]{amsmath}
%\usepackage{amsthm}
%\interdisplaylinepenalty=2500
%\savesymbol{iint}
%\usepackage{txfonts}
%\restoresymbol{TXF}{iint}
%\usepackage{wasysym}
\usepackage{amsthm}
%\usepackage{pifont}
%\usepackage{iithtlc}
% \usepackage{mathrsfs}
% \usepackage{txfonts}
 \usepackage{stfloats}
% \usepackage{steinmetz}
 \usepackage{bm}
% \usepackage{cite}
% \usepackage{cases}
% \usepackage{subfig}
%\usepackage{xtab}
\usepackage{longtable}
%\usepackage{multirow}
%\usepackage{algorithm}
%\usepackage{algpseudocode}
\usepackage{enumitem}
 \usepackage{mathtools}
 \usepackage{tikz}
% \usepackage{circuitikz}
% \usepackage{verbatim}
%\usepackage{tfrupee}
\usepackage[breaklinks=true]{hyperref}
%\usepackage{stmaryrd}
%\usepackage{tkz-euclide} % loads  TikZ and tkz-base
%\usetkzobj{all}
\usepackage{listings}
    \usepackage{color}                                            %%
    \usepackage{array}                                            %%
    \usepackage{longtable}                                        %%
    \usepackage{calc}                                             %%
    \usepackage{multirow}                                         %%
    \usepackage{hhline}                                           %%
    \usepackage{ifthen}                                           %%
  %optionally (for landscape tables embedded in another document): %%
    \usepackage{lscape}     
% \usepackage{multicol}
% \usepackage{chngcntr}
%\usepackage{enumerate}
\usepackage{tfrupee}

%\usepackage{wasysym}
%\newcounter{MYtempeqncnt}
\DeclareMathOperator*{\Res}{Res}
\DeclareMathOperator*{\equals}{=}


% correct bad hyphenation here
\hyphenation{op-tical net-works semi-conduc-tor}
\def\inputGnumericTable{}                                 %%

\lstset{
%language=C,
frame=single, 
breaklines=true,
columns=fullflexible
}
%\lstset{
%language=tex,
%frame=single, 
%breaklines=true
%}
\begin{document}

%


\newtheorem{theorem}{Theorem}[section]
\newtheorem{problem}{Problem}
\newtheorem{proposition}{Proposition}[section]
\newtheorem{lemma}{Lemma}[section]
\newtheorem{corollary}[theorem]{Corollary}
\newtheorem{example}{Example}[section]
\newtheorem{definition}[problem]{Definition}
\newcommand{\BEQA}{\begin{eqnarray}}
\newcommand{\EEQA}{\end{eqnarray}}
\newcommand{\define}{\stackrel{\triangle}{=}}
\newcommand*\circled[1]{\tikz[baseline=(char.base)]{
    \node[shape=circle,draw,inner sep=2pt] (char) {#1};}}
\bibliographystyle{IEEEtran}
\providecommand{\mbf}{\mathbf}
\providecommand{\pr}[1]{\ensuremath{\Pr\left(#1\right)}}
\providecommand{\qfunc}[1]{\ensuremath{Q\left(#1\right)}}
\providecommand{\sbrak}[1]{\ensuremath{{}\left[#1\right]}}
\providecommand{\lsbrak}[1]{\ensuremath{{}\left[#1\right.}}
\providecommand{\rsbrak}[1]{\ensuremath{{}\left.#1\right]}}
\providecommand{\brak}[1]{\ensuremath{\left(#1\right)}}
\providecommand{\lbrak}[1]{\ensuremath{\left(#1\right.}}
\providecommand{\rbrak}[1]{\ensuremath{\left.#1\right)}}
\providecommand{\cbrak}[1]{\ensuremath{\left\{#1\right\}}}
\providecommand{\lcbrak}[1]{\ensuremath{\left\{#1\right.}}
\providecommand{\rcbrak}[1]{\ensuremath{\left.#1\right\}}}
\theoremstyle{remark}
\newtheorem{rem}{Remark}
\newcommand{\sgn}{\mathop{\mathrm{sgn}}}
\providecommand{\fourier}{\overset{\mathcal{F}}{ \rightleftharpoons}}
%\providecommand{\hilbert}{\overset{\mathcal{H}}{ \rightleftharpoons}}
\providecommand{\system}{\overset{\mathcal{H}}{ \longleftrightarrow}}
	%\newcommand{\solution}[2]{\textbf{Solution:}{#1}}
\newcommand{\solution}{\noindent \textbf{Solution: }}
\newcommand{\cosec}{\,\text{cosec}\,}
\providecommand{\dec}[2]{\ensuremath{\overset{#1}{\underset{#2}{\gtrless}}}}
\newcommand{\myvec}[1]{\ensuremath{\begin{pmatrix}#1\end{pmatrix}}}
\newcommand{\mydet}[1]{\ensuremath{\begin{vmatrix}#1\end{vmatrix}}}
%\numberwithin{equation}{section}
%\numberwithin{figure}{section}
%\numberwithin{table}{section}
%\numberwithin{equation}{subsection}
%\numberwithin{problem}{section}
%\numberwithin{definition}{section}
\makeatletter
\@addtoreset{figure}{problem}
\makeatother
\let\StandardTheFigure\thefigure
\let\vec\mathbf
%\renewcommand{\thefigure}{\theproblem.\arabic{figure}}
\renewcommand{\thefigure}{\theproblem}
%\setlist[enumerate,1]{before=\renewcommand\theequation{\theenumi.\arabic{equation}}
%\counterwithin{equation}{enumi}
%\renewcommand{\theequation}{\arabic{subsection}.\arabic{equation}}
\def\putbox#1#2#3{\makebox[0in][l]{\makebox[#1][l]{}\raisebox{\baselineskip}[0in][0in]{\raisebox{#2}[0in][0in]{#3}}}}
     \def\rightbox#1{\makebox[0in][r]{#1}}
     \def\centbox#1{\makebox[0in]{#1}}
     \def\topbox#1{\raisebox{-\baselineskip}[0in][0in]{#1}}
     \def\midbox#1{\raisebox{-0.5\baselineskip}[0in][0in]{#1}}
\title{
	%\logo{
%Computational Approach to School Geometry
	Assignment
%	}
}
\author{ Burra Vishal Mathews\\CS21BTECH11010% <-this % stops a space
}
\maketitle
\tableofcontents
\bigskip
\renewcommand{\thefigure}{\theenumi}
\renewcommand{\thetable}{\theenumi}

\begin{abstract}
This manual provides solutions to the Assignment of Random Numbers
\end{abstract}
%template ends here
\section{Uniform Random Numbers}
Let $U$ be a uniform random variable between 0 and 1.
\begin{enumerate}[label=\thesection.\arabic*
,ref=\thesection.\theenumi]
\item Generate $10^6$ samples of $U$ using a C program and save into a file called uni.dat .
\\
\solution Download the following files and execute the  C program.
\begin{lstlisting}
wget https://github.com/VishalBurra27/randomvar/blob/main/URN/code/q1.1.c
wget https://github.com/VishalBurra27/randomvar/blob/main/URN/code/coeffs.h
\end{lstlisting}
Download the above files and execute the following commands
\begin{lstlisting}
$ gcc q1.1.c -lm
$ ./a.out
\end{lstlisting}
\item
Load the uni.dat file into python and plot the empirical CDF of $U$ using the samples in uni.dat. The CDF is defined as
\begin{align}
F_{U}(x) = \pr{U \le x}
\end{align}
\\
\solution  The following code plots Fig. \ref{fig:1.2}
\begin{lstlisting}
wget https://github.com/VishalBurra27/randomvar/blob/main/URN/code/q1.2.py
\end{lstlisting}
Download the above files and execute the following commands to produce Fig.\ref{fig:1.2}
\begin{lstlisting}
$ python3 q1.2.py
\end{lstlisting}
\begin{figure}[!h]
\centering
\includegraphics[width=\columnwidth]{./figs/q1.2.png}
\caption{The CDF of $U$}
\label{fig:1.2}
\end{figure}
%
\item
Find a  theoretical expression for $F_{U}(x)$.\\
\solution Given $U$ is a uniform Random Variable
\begin{align}
p_{U}(x)=1 \text{ for } \\
F_U(x)=\int_{-\infty}^{\infty}p_{U}(x)dx\\
\boxed{\implies F_U(x)=
\begin{cases}
 0 &x\le0\\
 x &0< x< 1\\
 1 &x\ge 1
\end{cases}}
\end{align}
\item
The mean of $U$ is defined as
%
\begin{equation}
E\sbrak{U} = \frac{1}{N}\sum_{i=1}^{N}U_i
\end{equation}
%
and its variance as
%
\begin{equation}
\text{var}\sbrak{U} = E\sbrak{U- E\sbrak{U}}^2 
\end{equation}
Write a C program to  find the mean and variance of $U$. \\
\solution Download the following files and execute the  C program.
\begin{lstlisting}
wget https://github.com/VishalBurra27/randomvar/blob/main/URN/code/q1.4.c
wget https://github.com/VishalBurra27/randomvar/blob/main/CLT/code/coeffs.h
\end{lstlisting}
Download the above files and execute the following commands
\begin{lstlisting}
$ gcc q1.4.c -lm
$ ./a.out
\end{lstlisting}
\item Verify your result theoretically given that
\end{enumerate}
%
\begin{equation}
E\sbrak{U^k} = \int_{-\infty}^{\infty}x^kdF_{U}(x)
\end{equation}
\solution 
\begin{align}
    \text{var}\sbrak{U} &= E\sbrak{U- E\sbrak{U}}^2\\ 
    \implies \text{var}\sbrak{U} &= E\sbrak{U^2}- E\sbrak{U}^2 \\
    E\sbrak{U}&=\int_{-\infty}^{\infty}xdF_U(x)\\
    E\sbrak{U}&=\int_{0}^{1}x\\
    \implies \boxed{E\sbrak{U}=\frac{1}{2}}\\
    E\sbrak{U^2}&=\int_{-\infty}^{\infty}x^{2}dF_U(x)\\
    E\sbrak{U^2}&=\int_{0}^{1}x^{2}dF_U(x)\\
    \implies E\sbrak{U^2}&=\frac{1}{3}\\
    \implies \boxed{\text{var}\sbrak{U}=\frac{1}{12}=0.0833}
\end{align}
\section{Central Limit Theorem}
%
\begin{enumerate}[label=\thesection.\arabic*
,ref=\thesection.\theenumi]
%
\item
Generate $10^6$ samples of the random variable
%
\begin{equation}
X = \sum_{i=1}^{12}U_i -6
\end{equation}
%
using a C program, where $U_i, i = 1,2,\dots, 12$ are  a set of independent uniform random variables between 0 and 1 and save in a file called gau.dat\\
\solution Download the following files and execute the  C program.
\begin{lstlisting}
wget https://github.com/VishalBurra27/randomvar/blob/main/CLT/code/q2.1.c
wget https://github.com/VishalBurra27/randomvar/blob/main/CLT/code/coeffs.h
\end{lstlisting}
Download the above files and execute the following commands
\begin{lstlisting}
$ gcc q2.1.c -lm
$ ./a.out
\end{lstlisting}
\item
Load gau.dat in python and plot the empirical CDF of $X$ using the samples in gau.dat. What properties does a CDF have?\\
\solution The CDF of $X$ is plotted in Fig. \ref{fig:2.2}\\
using the code below
\begin{lstlisting}
wget https://github.com/VishalBurra27/randomvar/blob/main/CLT/code/q2.2.py
\end{lstlisting}
Download the above files and execute the following commands to produce Fig.\ref{fig:2.2}
\begin{lstlisting}
$ python3 q2.2.py
\end{lstlisting}
\begin{figure}[!h]
\centering
\includegraphics[width=\columnwidth]{./figs/q2.2.png}
\caption{The CDF of $X$}
\label{fig:2.2}
\end{figure}
Some of the properties of CDF 
\begin{enumerate}
\item $\lim_{x \to \infty}F_X(x) = 1$
    \item $F_X(x)$ is non decreasing function.
    \item Symmetric about one point.
\end{enumerate}
\item
Load gau.dat in python and plot the empirical PDF of $X$ using the samples in gau.dat. The PDF of $X$ is defined as
\begin{align}
p_{X}(x) = \frac{d}{dx}F_{X}(x)
\end{align}
What properties does the PDF have?
\\
\solution The PDF of $X$ is plotted in Fig. \ref{fig:2.3} using the code below
\begin{lstlisting}
wget https://github.com/VishalBurra27/randomvar/blob/main/CLT/code/q2.3.py
\end{lstlisting}
Download the above files and execute the following commands to produce Fig.\ref{fig:2.3}
\begin{lstlisting}
$ python3 q2.3.py
\end{lstlisting}
\begin{figure}[!h]
\centering
\includegraphics[width=\columnwidth]{./figs/q2.3.png}
\caption{The PDF of $X$}
\label{fig:2.3}
\end{figure}
Some of the properties of the PDF:
\begin{enumerate}
    \item Symmetric about $x=\mu$ in this case
    \item Decreasing function for $x>\mu$ and increasing for $x<\mu$ and attains maximum at $x=\mu$
    \item Area under the curve is unity.
\end{enumerate}
\item Find the mean and variance of $X$ by writing a C program.\\
\solution Download the following files and execute the  C program.
\begin{lstlisting}
wget https://github.com/VishalBurra27/randomvar/blob/main/CLT/code/q2.4.c
wget https://github.com/VishalBurra27/randomvar/blob/main/CLT/code/coeffs.h
\end{lstlisting}
Download the above files and execute the following commands
\begin{lstlisting}
$ gcc q2.4.c -lm
$ ./a.out
\end{lstlisting}
\item Given that 
\begin{align}
p_{X}(x) = \frac{1}{\sqrt{2\pi}}\exp\brak{-\frac{x^2}{2}}, -\infty < x < \infty,
\end{align}
repeat the above exercise theoretically.
\end{enumerate}
\solution 
\begin{enumerate}
    \item CDF is given by 
    \begin{align}
        F_X(x)&=\int_{-\infty}^{\infty}p_X(x)dx\\
        \boxed{F_X(x)=1}
    \end{align}
    \item Mean is given by
    \begin{align}
        E(x)=\int_{-\infty}^{\infty}xp_X(x)dx\\
        \implies \boxed{E(x)=0}
    \end{align}
    \item Variance is given by
    \begin{align}
        \text{var}\sbrak{U}=E(U^2)-(E(U))^2\\
        \implies\boxed{\text{var}\sbrak{U}=\sqrt{2}}
    \end{align}
\end{enumerate}
\section{From Uniform to Other}
\begin{enumerate}[label=\thesection.\arabic*
,ref=\thesection.\theenumi]
%
\item
Generate samples of 
%
\begin{equation}
V = -2\ln\brak{1-U}
\end{equation}
%
and plot its CDF.  \\
\solution Download the following files and execute the  C program.
\begin{lstlisting}
wget https://github.com/VishalBurra27/randomvar/blob/main/FUO/code/q3.1.c
wget https://github.com/VishalBurra27/randomvar/blob/main/FUO/code/coeffs.h
\end{lstlisting}
Download the above files and execute the following commands
\begin{lstlisting}
$ gcc q3.1.c -lm
$ ./a.out
\end{lstlisting}
The CDF of $V$ is plotted in Fig. \ref{fig:3.1} using the code below
\begin{lstlisting}
wget https://github.com/VishalBurra27/randomvar/blob/main/FUO/code/q3.1.py
\end{lstlisting}
Download the above files and execute the following commands to produce Fig.\ref{fig:3.1}
\begin{lstlisting}
$ python3 q3.1.py
\end{lstlisting}
\begin{figure}[!h]
\centering
\includegraphics[width=\columnwidth]{./figs/q3.1.png}
\caption{The CDF of $X$}
\label{fig:3.1}
\end{figure}
\item Find a theoretical expression for $F_V(x)$.\\
\solution
If Y = g(X), we know that $F_Y(y) = F_X(g^{-1}(y))$, here 
\begin{align}
V &= -2\ln{(1-U)} \\
1-U &= e^{\frac{-V}{2}}\\
U &= 1 - e^{\frac{-V}{2}} \\ 
F_V(x) &= F_U(1 - e^{\frac{-x}{2}}) 
\end{align}
 \begin{align}
\implies
  F_V(x)=
  \begin{cases}
   0                         & x < 0 \\
	1 - e^{\frac{-x}{2}} & x \geq 0
	\end{cases}
 \end{align}
%\item
%Generate the Rayleigh distribution from Uniform. Verify your result through graphical plots.
\end{enumerate}
\end{document}
